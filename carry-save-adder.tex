\documentclass[11pt, twoside]{article}
\usepackage{graphicx}



\begin{document}
\begin{center}


\title{\begin{huge}\textbf{A survey report on carry save adder}\end{huge}}
\maketitle

\author{\begin{large}\textit{Ritika Kumari}\end{large}}

\title{\begin{large}\textit{Roll no. 39/CSE/17069/267}\end{large}}
\maketitle

\title{\begin{large}\textit{Email id: ritikakumari1302@gmail.com}\end{large}}
\maketitle


\end{center}

\begin{abstract}

This paper presents a technology-independent design and simulation of a modified architecture of the Carry-Save Adder. This architecture is shown to produce the result of the addition fast and by requiring a minimum number of logic gates. Binary addition is carried out by a series of XOR, AND and Shift-left operations. These operations are terminated with a completion signal indicating that the result of the addition is obtained. Because the number of shift operations carried out varies from 0 to n for n-bit addends, a behavioral model was developed in which all the possible addends having 2- to 15-bits were applied. A mathematical model was deducted from the data and used to predict the average number of shift required for standard binary numbers such as 32, 64 or 128-bits. 4-bit prototypes of this adder were designed and simulated in both synchronous and asynchronous modes of operation.


\end{abstract}

\section{Introduction}
n digital computer arithmetic, addition and subtraction are the most basic core operations, especially for 
digital  signal  processing  applications.  The  other  two  fundamental  operations  li
ke  multiplication  and 
division  are  too  performed  using  addition  and  subtraction  and  hence  they  play  a  very  important  role  in 
processors like microprocessors, microcontrollers and digital signal processors. The fast adders can be 
used to speed up the ar
ithmetic operations in a processor. There are numerous ways available to perform 
addition but with different trade
-
offs. Some are good in producing low power at the cost of area and some 
are best in offering performance (i.e., high speed) at the cost of po
wer and/or area.
Arithmetic  operations  are  essential  building  blocks  in  any  system;  either  the  system  can  be  designed 
based  on  a  processor,  or  an  FPGA/ASIC.  The  data  path  circuitries  in  a  microprocessor,  Multiply
-
Accumulate   (MAC)   operations   in   a 
digital   signal   processor   and   high   speed   integrated   circuits   in 
communication  systems  are  the  few  application  examples  which  would  perform  arithmetic  operations 
especially  using  regular  full  adders.  The  k
-
bit  ripple
-
carry  adder  is  the  most  simplest  adde
r  structure, 
which adds  two  words having  k
-
bits.  The  delay  of  the  RCA  is  very  high since  the  carry  is  propagated  (i.e., 
rippled) to the full adder stages to produce a final sum. If the value of k is very large, then the delay would 
be  more.  These  RCAs  are 
not  suitable  in  situations  where  the  speed  data  processing  is  involved.  So  the 
demand  to  design  fast  adders  grows  rapidly  to  meet  the  current  high  speed  integrated  circuits  trend 
CSA  is  a  kind  of  adder  with  low  propagation  delay,  but  instead  of  adding  two  input  numbers  to  a  single  sum 
output,  it adds three input  numbers to an output pair of  numbers. When its two outputs  are  then summed by a 
traditional  carry
-
lookahead  or
ripple  carry  adder
,  we  received  the  sum  of  all  three  inputs.  In  particular,  the 
propagation delay of a CSA is not affected by the width of vectors being added. Each full adder‟s output S is 
connected to corresponding output bit of one output, and it‟s out
put Cout is connected to the next higher output 
bit of the second output; the lowest bit of the second output is fed directly from the carry
-
save‟s Cin input

The carry save addition of 2 N-bit numbers results in two (N + 1)-bit numbers being produced, the virtual carry (VC) and the virtual sum (VS).But after getting VC and VS you still have to add the two values together with a convectional adder to get your final result, so only adding 2 numbers is pointless.Take this example, lets say one carry-save addition takes k*T ms, where k = number of N - bit numbers being added, and a convectional adder takes 5T ms to add 2 numbers (regardless of bit width), then if:

1) you added 2 numbers, then

$Time(Carry-Save) = 2T + 5T = 7T$

$Time(Convectional Adder) = 5T$

2) you added 3 numbers, then

$Time(Carry-Save) = 3T + 5T = 8T$

$*Time(Convectional Adder) = 5T + 5T = 10T$

So carry-save addition is only useful if you have at least 3 operands to add.

\includegraphics[scale=0.3]{image1.png}

\section{MULTI-OPERAND ADDITION USING CARRY SAVE ADDERS}


The  addition  of  more  than  two numbers would  be  done using carry
-
save adders  based  on 
the  fact  that a 
full
-
adder has 3 inputs and obtains 2 outputs and hence the full adder becomes the basic building block .  The  addition  of  four  4
-
bit  numbers,  four  8
-
bit  numbers,  four  16
-
bit  numbers,  four  32
-
bit  numbers 
and four 64
-
bit numbers are ex
emplified using CSA principle in this study.
Addition of four 4
-
bit numbers
A
0
-
3
,  B
0
-
3
,  C
0
-
3
and  D
0
-
3
are  the  four  4
-
bit  numbers  which  are  used  for  addition.
.  The  carry  from  each 
stage is propagated down instead in the same stage and hence reduce the overall delay. These are known 
as “carry save” stages and varies depend on the number of operands. Finally, the ripple
-
carry  adder  is 
used  to  obta
in  the  final  sum  at  the  cost  of  delay.\\

\includegraphics[scale=0.3]{image2.png}

\section{FINAL SUM COMPUTATION METHODS OF CSA}
Multi
-
operand addition using CSA has two important stages as “carry
-
save stages” and “final stage” to 
compute the overall sum. This section exemplifies the final stage using ripple
-
carry adder, carry look
-
ahead 
adder and carry select adder to determine
the final output sum.
Sum output using RCA 
In this study, the final sum output using RCA is referred in this study as “CSA
-
RCA”. The carry from the 
previous  stages  are  considered  here  and  propagated  within  the  last  stage  to  find  the  final  sum  output 
valu
e. 

Sum output using CLA
The last stage is replaced with carry look
-
ahead adder circuit and is referred to as “CSA
-
CLA”. The carry out 
is calculated in advance based on the “propagate” and “generate” terms instead of depending on the 
previous ca
rry. Since the rippling effect is reduced, this circuit should offer better performance compared 
with  the  RCA  combination.
.  The  VHDL  portion  of    a  CLA 
stage with the addition of four 8
-
bit numbers is shown i
n Figure 11. If a
0
= b
0
= 1 (i.e., g
0
= 1), then the c1 
is computed as ‘1’ based on the 
generate
term. If p
0
= 1 and c
0
= 1, then the c1 is computed as ‘1’ based 
on the 
propagate
term. The CLA circuit is constructed based on the following equation (3). 
C
i
+1
= (A
i

B
i
)C
i
+ A
i
B
i
= P
i
C
i
+ G
i
(3)
For a 4
-
bit CLA circuit, these carry terms can be computed as,

\includegraphics[scale=0.3]{image3.png}
\section{References}
 \subsection{ 1. FPGA IMPLEMENATION OF HIGH SPEED AND LOWPOWER CARRY SAVE ADDER 
VS.Balaji , Har Narayan Upadhyay}

 \subsection{ 2. https://www.quora.com/How-should-I-design-a-carry-save-adder-circuit-so-that-I-can-make-it-as-fast-and-compact-as-possible}
  \subsection{3. http://www.geoffknagge.com/fyp/carrysave.shtml}
 \subsection{4. Behrooz Parhami, "Computer Arithmetic", Oxford Press, 2000, pp131}
 \subsection{5. Performance Analysisof Different Bit  Carry Look Ahead Adder Using VHDL  Environment
Rajender Kumar,Sandeep DahiyaSES,BPSMV,Khanpur Kalan, Gohan, Sonipat ,Haryana
 International Journal of Engineering Science and Innovative Technology (IJESIT)
Volume 2, Issue 
4, July 2013}
 \subsection{6. Modified energy efficient carry save adder
 Benisha bennet, s. maflin
 2015 International Conference on Circuits, Power and Computing Technologies [ICCPCT-2015]}
\subsection{7.  Design of carry save adder using transmission gate logic
 J.Princy Joice 
Dept of ECE/Sathyabama University, Chennai/Tamilnadu/India 
M.Anitha
Dept of ECE/Sathyabama University, Chennai/Tamilnadu/India 
Mrs.I.Rexlin Sheeba
Assistant Professor, Dept of ECE/Sathyabama University, Chennai/Tamilnadu/India
 INTERNATIONAL JOURNAL OF INNOVATIONS IN ENGINEERING
 RESEARCH AND TECHNOLOGY [IJIERT] 
ISSN: 
2394-3696 
VOLUME 2, ISSUE 1 JAN-2015}

\end{document}

